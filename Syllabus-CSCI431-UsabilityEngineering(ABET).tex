\documentclass[12pt]{article}

\newlength\tindent
\setlength{\tindent}{\parindent}
\setlength{\parindent}{0pt}
\renewcommand{\indent}{\hspace*{\tindent}}

\title{Syllabus: CSCI 431 Software Engineering Tools}
\author{Kevin Buffardi}
\date{February 2015}
\begin{document}
  \maketitle
  \subsection*{Overview}	 
  \noindent
  This course is an in-depth study of user experience (UX) design with an emphasis on usability evaluation methods. Students practice hands-on techniques including: usability testing, survey design, card sorting, contextual inquiry, wireframing and rapid prototyping. Students will work in interdisciplinary teams on user experience design projects.
  \newline
  \underline{Prerequisites}: CINS 110, CINS 465, or CSCI 430 for CSCI/CINS majors; CDES 322, CDES 327, CDES 437 or APCG 360 for other majors. \\
  This course is required for the Computer Science BS major and is worth 3 credits. 
  
  \subsection*{Instructor}
  \noindent
  Kevin Buffardi, kbuffardi@csuchico.edu, Office hours: OCNL 220 Tues/Thurs 10-12. \\

  \subsection*{Required Materials}
  \noindent
  Interaction Design: Beyond Human-Computer Interaction (2nd Edition or later) \\
  Yvonne Rogers, Helen Sharp, Jenny Preece \\
  There will also be selected readings, provided by the instructor \\
  \newline
  Laptop computer (Mac OSX or Windows 7+ preferred)

  \subsection*{Learning Outcomes}
  \noindent
  By completing this course, students will be able to: \\
    Understand and explain the purpose and value of user-centered design; Demonstrate user inquiry methods (contextual inquiry, surveys, card sorting); Analyze results of user inquiry methods; Create low-fidelity user interface wireframes; Plan and develop prototypes for rapid-development; Create research design for a usability study; Create a recruitment and user test/interview protocol; Demonstrate fundamentals of conducting a user test/interview; Analyze findings from a user study; Create a report of usability findings and recommendations \\

  The course addresses following broader CSCI Learning Outcomes: \\
  \textit{\textbf{I}ntroduces, \textbf{P}racticed, \textbf{A}ssessed} \\
    a. An ability to apply knowledge of computing and mathematics appropriate to the discipline. \textbf{(P)}; b.  An ability to analyze a problem, and identify and define the computing requirements appropriate to its solution. \textbf{(P)}; c. An ability to design, implement, and evaluate a computer-based system, process, component, or program to meet desired needs. \textbf{(A)}; d. An ability to function effectively on teams to accomplish a common goal. \textbf{(A)}; e. An understanding of professional, ethical, legal, security and social issues and responsibilities. \textbf{(A)}; f. An ability to communicate effectively with a range of audiences. \textbf{(A)}; g. An ability to analyze the local and global impact of computing on individuals, organizations, and society. \textbf{(A)}; h. Recognition of the need for and an ability to engage in continuing professional development. \textbf{(P)}; i. An ability to use current techniques, skills, and tools necessary for computing practice. \textbf{(A)}; j. An ability to apply mathematical foundations, algorithmic principles, and computer science theory in the modeling and design of computer-based systems in a way that demonstrates comprehension of the tradeoffs involved in design choices. \textbf{(P)}; k. An ability to apply design and development principles in the construction of software systems of varying complexity. \textbf{(P)}

  \subsection*{Topics}
  \noindent
  Topics for this class include: usability, user inquiry methods, wireframing, rapid prototype development, usability evaluation methods
  \end{document}