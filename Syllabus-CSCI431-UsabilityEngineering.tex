\documentclass[12pt]{article}

\newlength\tindent
\setlength{\tindent}{\parindent}
\setlength{\parindent}{0pt}
\usepackage{parskip}
\renewcommand{\indent}{\hspace*{\tindent}}

\title{Syllabus: CSCI 431 Usability Engineering}
\author{Kevin Buffardi}
\date{October 2019}
\begin{document}
  \maketitle
  \section*{Overview}	 
  \noindent
  Usability Engineering is the craft of designing and evaluating how people\\
  interact with digital interfaces. In this class, students learn and apply\\
  the fundamentals of user experience (UX) design and human-computer\\
  interaction (HCI) principles. Students engage in team projects and conduct\\
  user research to design, evaluate, and revise interactive prototypes. 2\\
  hours discussion, 2 hours activity.
  \newline
  \underline{Prerequisites}: Junior Standing.
  
  \section*{Instructor}
  \noindent
  Kevin Buffardi
  \newline
  kbuffardi@csuchico.edu
  \newline
  OCNL 220
  \newline
  Office Hours: TBD

  \section*{Required Materials}
  \noindent
  No required textbook; free reading materials will be assigned and provided in class.
  \newline
  A laptop or tablet computer is required. Mac OSX or Windows 7+ is preferred for some software we will use. *nix may be suitable as well, but mobile-specific operating systems like Android, iOS, or ChromeOS are insufficient. No software purchases required. Bring the laptop to both lecture and lab, prepared with a charged battery (and/or your charger).
  \newline
  Low-cost, required materials include:
  \begin{itemize}
    \item Sketch book/pad (white paper, unlined, at least 8.5x11")
    \item Graphite drawing pencil (black color, non-mechanical)
    \item Eraser
  \end{itemize}

  Students are responsible for all announcements as well as for taking their own notes in class. Absences should be pre-excused with the instructor and notes should be borrowed from a trusted classmate.

  \textit{Note paper and writing utensils are necessary for notetaking}

  \section*{Learning Outcomes}
  \noindent
  By completing this course, students will be able to:
  \begin{itemize}
    \item Understand usability goals and explain how they impact user experience
    \item Understand and follow the process of user-centered design
    \item Explain the value of following user-centered design methods
    \item Demonstrate user inquiry methods (e.g. contextual inquiry, surveys, card sorting)
    \item Analyze results of user inquiry methods
    \item Design low-fidelity user interface wireframes
    \item Plan and develop prototypes for rapid-development
    \item Create research design for a usability study 
    \item Create a recruitment and user test/interview protocol
    \item Demonstrate fundamentals of conducting a user test/interview
    \item Analyze findings from a user study
    \item Create a report of usability findings and recommendations
    \item Create an online portfolio that communicates their work to different audiences
  \end{itemize}

  This course fulfills the Upper Division Pathway Social Sciences (UD-D) disciplinary area. In this class, students will satisfy the General Education Student Learning Outcomes (SLO):
  \begin{itemize}
    \item \textbf{Critical Thinking}\\
     Identifies issues and problems in human-computer interteraction by\\
     analyzing and assessing their designs, according to usability goals.
    \item \textbf{Active Inquiry}\\
     Demonstrates knowledge of usability evaluation by applying\\
     appropriate research methods and interpreting how their findings relate \\
     to human cognition and behavior
    \item \textbf{Creativity}\\
     Takes intellectual risks and applies novel approaches to designing\\
     unique solutions for user interactions.
  \end{itemize}

  This course is also an upper-division Writing (W) course. Accordingly, the course will include:
  \begin{itemize}
    \item \textbf{Written Communication}\\
     Students will practice writing to communicate usability requirements
     to designers and developers as well as summarizing research findings for
     business stakeholders. Students will also write journal entries to
     communicate to broad audiences about real-life impacts of usability.
    \item \textbf{Critique and Revision}\\
     Students will revise written assignments to reduce errors in
     grammar/syntax/punction/spelling and to improve writing techniques
     within the discipline. Students will also critique other students'
     writing in guided peer-review activities. 
  \end{itemize}

  \section*{Assessment}
  \noindent
  Grades for this course will be determined by the following assessments, with the provided weights:
  \begin{itemize}
    \item Team Project Deliverables \textbf{(50\%)}, including:\hfill
      \begin{itemize}
        \item User Inquiry and results 
        \item Wireframes
        \item Initial Prototype
        \item User Study Protocols
        \item Usability Report \& Portfolio
        \item Team \& Peer Evaluation
      \end{itemize}
    \item Individual Assignments:\hfill
      \begin{itemize}
        \item In-class exercises \textbf{(20\%, cumulative)}
        \item Progress checks (periodic, short quizzes) \textbf{(20\%, cumulative)}
        \item Portfolio journal \textbf{(8\%)}
        \item User introduction \textbf{(2\%)}
      \end{itemize}
  \end{itemize}

  Letter grades are assigned on a 10-point scale per letter; the top 2 points for each letter earn a plus (+) designation while the bottom 2 points earn a minus (-). For example, 89-89 earns a B+ while 82-87 earns a B and 80-81 earns a B-. Fractions are rounded to the closest whole number.
  
  \section*{Schedule}
  \noindent
  The following is a \underline{tentative schedule}:\\
  \textbf{Week} - Topic 
  \begin{enumerate}
    \item - User eXperience (UX): Overview and Purpose
    \item - User Inspection Methods \& Ethics in User Research
    \item - Contextual Inquriy \& Task Analysis
    \item - Card Sorting \& Information Architecture
    \item - User Personas \& Storyboards
    \item - Wireframing
    \item - Writing user experience guidelines
    \item - Low-Fidelity prototype design
    \item - High-Fidelity prototype implementation
    \item - User study research design \& iteration
    \item - Recruitment and User Test protocol design
    \item - Conducting and Moderating User Tests
    \item - Analyzing qualitative and quantitative data
    \item - Interpreting usability recommendations
    \item - Writing user study report
  \end{enumerate}

  \section*{Accommodations}
  \noindent
  If you require any auxiliary aids, services, or other accommodations for this class, please identify your needs to your instructor by the end of the first week of class (via email or schedule an appointment in person), or as soon as you have the required documentation.
  \par
  If you wish to have special accommodations due to religious holidays, please request those accommodations by the end of the second week of class. Requests made after these deadlines may not be possible to honor. 

  \section*{Principles}
  \noindent
  Students in this class are encouraged to speak up and participate during class meetings. Because the class will represent a diversity of individual beliefs, backgrounds, and experiences, every member of this class must show respect for every other member of this class.
  \par
  I am part of the Safe Zone Ally community network of trained Chico State faculty/staff/students who are available to listen and support you in a safe and confidential manner. As a Safe Zone Ally, I can help you connect with resources on campus to address problems you may face that interfere with your academic and social success on campus as it relates to issues surrounding sexual orientation/gender identity. My goal is to help you be successful and to maintain a safe and equitable campus.

\end{document}
