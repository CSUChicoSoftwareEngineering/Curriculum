\documentclass[12pt]{article}

\newlength\tindent
\setlength{\tindent}{\parindent}
\setlength{\parindent}{0pt}
\renewcommand{\indent}{\hspace*{\tindent}}

\title{Syllabus: CSCI 431 Usability Engineering}
\author{Kevin Buffardi}
\date{November 2015}
\begin{document}
  \maketitle
  \section*{Overview}	 
  \noindent
  A study of user experience (UX) with an emphasis on user-centered design and usability evaluation methods. Students practice hands-on techniques including: usability testing, survey design, card sorting, contextual inquiry, wireframing and rapid prototyping. Students work in multi-disciplinary teams on user experience design projects.
  \newline
  \newline
  \underline{Prerequisites}: Junior Standing; CINS 110 with a grade of C- or higher for CSCI/CINS majors, or HTML/CSS/JavaScript experience for other majors. \\
  \newline
  The course is \underline{not} programming-intensive, however, it teaches interaction prototype development (using HTML, CSS, and JavaScript). Some basic familiarity with this technology will prepare students to further develop and apply these skills. 
  
  \section*{Instructor}
  \noindent
  Kevin Buffardi \\
  kbuffardi@csuchico.edu \\
  OCNL 220 \\
  Office Hours: TBD\\

  \section*{Required Materials}
  \noindent
  No required textbook; reading materials will be assigned in class.
  \newline
  Laptop computer (Mac OSX or Windows 7+ preferred)

  \section*{Learning Outcomes}
  \noindent
  By completing this course, students will be able to:
  \begin{itemize}
    \item Understand and explain the purpose and value of user-centered design
    \item Demonstrate user inquiry methods (contextual inquiry, surveys, card sorting)
    \item Analyze results of user inquiry methods
    \item Create low-fidelity user interface wireframes
    \item Plan and develop prototypes for rapid-development
    \item Create research design for a usability study 
    \item Create a recruitment and user test/interview protocol
    \item Demonstrate fundamentals of conducting a user test/interview
    \item Analyze findings from a user study
    \item Create a report of usability findings and recommendations
  \end{itemize}

  The course addresses following broader CSCI Learning Outcomes: \\
  \textit{\textbf{I}ntroduces, \textbf{P}racticed, \textbf{A}ssessed} \\
  \begin{itemize}
    \item[a.] An ability to apply knowledge of computing and mathematics appropriate to the discipline. \textbf{(P)}
    \item[b.]  An ability to analyze a problem, and identify and define the computing requirements appropriate to its solution. \textbf{(P)}
    \item[c.] An ability to design, implement, and evaluate a computer-based system, process, component, or program to meet desired needs. \textbf{(A)}
    \item[d.] An ability to function effectively on teams to accomplish a common goal. \textbf{(A)}
    \item[e.] An understanding of professional, ethical, legal, security and social issues and responsibilities. \textbf{(A)}
    \item[f.] An ability to communicate effectively with a range of audiences. \textbf{(A)}
    \item[g.] An ability to analyze the local and global impact of computing on individuals, organizations, and society. \textbf{(A)}
    \item[h.] Recognition of the need for and an ability to engage in continuing professional development. \textbf{(P)}
    \item[i.] An ability to use current techniques, skills, and tools necessary for computing practice. \textbf{(A)}
    \item[j.] An ability to apply mathematical foundations, algorithmic principles, and computer science theory in the modeling and design of computer-based systems in a way that demonstrates comprehension of the tradeoffs involved in design choices. \textbf{(P)} 
    \item[k.] An ability to apply design and development principles in the construction of software systems of varying complexity. \textbf{(P)}
  \end{itemize}

  \section*{Assessment}
  \noindent
  Grades for this course will be determined by the following assessments, with the provided weights:
  \begin{itemize}
    \item Team Project Deliverables \hfill \\
      \begin{itemize}
        \item User Inquiry and results \textbf{(5\%)}
        \item Wireframes \textbf{(5\%)}
        \item Initial Prototype \textbf{(10\%)}
        \item User Study Protocols \textbf{(10\%)}
        \item Usability Report \textbf{(20\%)}
        \item Team \& Peer Evaluation \textbf{(10\%)}
      \end{itemize}
    \item Individual Assignments: \hfill \\
      \begin{itemize}
        \item In-class exercises \textbf{(20\%, cumulative)}
        \item Progress checks (periodic, short quizzes) \textbf{(20\%, cumulative)}
      \end{itemize}
  \end{itemize}
  
  \section*{Schedule}
  \noindent
  The following is a \underline{rough schedule} of the semester, which is \underline{subject to change} based on the instructor's discretion: \\
  \newline
  \textbf{Week} - Topic 
  \begin{enumerate}
    \item - User eXperience: Overview and Purpose
    \item - Contextual Inquiry
    \item - Survey Design and Analysis
    \item - Card Sorting and Information Architecture
    \item - Wireframing
    \item - Introduction to prototyping
    \item - Rapid prototype design and iteration
    \item - Review of topics covered
    \item - Introduction to usability research design
    \item - Recruiting and User Test protocol design
    \item - Conducting and Moderating User Tests
    \item - Analyzing study data
    \item - Usability recommendations
    \item - Writing usability study report
    \item - Comprehensive course review
  \end{enumerate}

  \section*{Accommodations}
  \noindent
  If you require any auxiliary aids, services, or other accommodations for this class, please identify your needs to your instructor by the end of the first week of class (via email or schedule an appointment in person), or as soon as you have the required documentation.\\ 
  If you wish to have special accommodations due to religious holidays, please request those accommodations by the end of the second week of class. Requests made after these deadlines may not be possible to honor. 

  \section*{Principles}
  \noindent
  Respect: Students in this class are encouraged to speak up and participate during class meetings. Because the class will represent a diversity of individual beliefs, backgrounds, and experiences, every member of this class must show respect for every other member of this class. \\
  \newline
  I am part of the Safe Zone Ally community network of trained Chico State faculty/staff/students who are available to listen and support you in a safe and confidential manner. As a Safe Zone Ally, I can help you connect with resources on campus to address problems you may face that interfere with your academic and social success on campus as it relates to issues surrounding sexual orientation/gender identity. My goal is to help you be successful and to maintain a safe and equitable campus.

\end{document}
