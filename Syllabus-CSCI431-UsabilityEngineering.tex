\documentclass[12pt]{article}

\newlength\tindent
\setlength{\tindent}{\parindent}
\setlength{\parindent}{0pt}
\usepackage{parskip}
\renewcommand{\indent}{\hspace*{\tindent}}

\title{Syllabus: CSCI 431 Usability Engineering}
\author{Kevin Buffardi}
\date{October 2018}
\begin{document}
  \maketitle
  \section*{Overview}	 
  \noindent
  A study of designing and evaluating how people interact with computers and digital interfaces by introducing topics in user experience (UX) and human-computer interaction (HCI). 
  Students learn user-centered design and evaluation methods with hands-on, interdisciplinary projects.
  Skills and techniques taught include: usability testing, survey design, card sorting, contextual inquiry, wireframing, rapid prototyping, and digital interface design.
  \underline{Prerequisites}: Junior Standing.
  
  \section*{Instructor}
  \noindent
  Kevin Buffardi
  \newline
  kbuffardi@csuchico.edu
  \newline
  OCNL 220
  \newline
  Office Hours: TBD

  \section*{Required Materials}
  \noindent
  No required textbook; reading materials will be assigned in class.
  \newline
  A laptop or tablet computer is required. Mac OSX or Windows 7+ is preferred for some software we will use. *nix may be suitable as well, but mobile-specific operating systems like Android, iOS, or ChromeOS are insufficient. No software purchases required. Bring the laptop to both lecture and lab, prepared with a charged battery (and/or your charger).
  \newline
  Other required materials include:
  \begin{itemize}
    \item Sketch book/pad (white paper, unlined, at least 8.5x11")
    \item Graphite drawing pencil (black color, non-mechanical)
    \item Eraser
  \end{itemize}

  \section*{Learning Outcomes}
  \noindent
  By completing this course, students will be able to:
  \begin{itemize}
    \item Understand usability goals and explain how they impact user experience
    \item Understand and follow the process of user-centered design
    \item Explain the value of following user-centered design methods
    \item Demonstrate user inquiry methods (contextual inquiry, surveys, card sorting)
    \item Analyze results of user inquiry methods
    \item Design low-fidelity user interface wireframes
    \item Plan and develop prototypes for rapid-development
    \item Create research design for a usability study 
    \item Create a recruitment and user test/interview protocol
    \item Demonstrate fundamentals of conducting a user test/interview
    \item Analyze findings from a user study
    \item Create a report of usability findings and recommendations
  \end{itemize}

  This course fulfills the Upper Division Pathway Arts/Humanities (UD-C) disciplinary area. In this class, students will satisfy the General Education Student Learning Outcomes (SLO):
  \begin{itemize}
    \item \textbf{Critical Thinking}\\
     Identifying issues and problems in human-computer interfaces by\\
     analyzing and assessing their design according to usability goals.
    \item \textbf{Active Inquiry}\\
     Demonstrates knowledge of and applies research techniques to\\
     evaluating usability and how it relates to principles of human cognition\\
     and behavior
    \item \textbf{Creativity}\\
    Takes intellectual risks and applies novel approaches to designing\\
    unique solutions for user interactions.
  \end{itemize}

  \section*{Assessment}
  \noindent
  Grades for this course will be determined by the following assessments, with the provided weights:
  \begin{itemize}
    \item Team Project Deliverables \textbf{(50\%)}, including:\hfill
      \begin{itemize}
        \item User Inquiry and results 
        \item Wireframes
        \item Initial Prototype
        \item User Study Protocols
        \item Usability Report
        \item Team \& Peer Evaluation
      \end{itemize}
    \item Individual Assignments:\hfill
      \begin{itemize}
        \item In-class exercises \textbf{(20\%, cumulative)}
        \item Progress checks (periodic, short quizzes) \textbf{(20\%, cumulative)}
        \item Class and message board participation \textbf{(8\%)}
        \item User introduction \textbf{(2\%)}
      \end{itemize}
  \end{itemize}

  Letter grades are assigned on a 10-point scale per letter; the top 2 points for each letter earn a plus (+) designation while the bottom 2 points earn a minus (-). For example, 89-89 earns a B+ while 82-87 earns a B and 80-81 earns a B-. Fractions are rounded to the closest whole number.
  
  \section*{Schedule}
  \noindent
  The following is a \underline{tentative schedule}:\\
  \textbf{Week} - Topic 
  \begin{enumerate}
    \item - User eXperience: Overview and Purpose
    \item - Contextual Inquiry
    \item - Survey Design and Analysis
    \item - Card Sorting and Information Architecture
    \item - Wireframing
    \item - Introduction to prototyping
    \item - Rapid prototype design and iteration
    \item - Review of topics covered
    \item - Introduction to usability research design
    \item - Recruiting and User Test protocol design
    \item - Conducting and Moderating User Tests
    \item - Analyzing study data
    \item - Usability recommendations
    \item - Writing usability study report
    \item - Comprehensive course review
  \end{enumerate}

  \section*{Accommodations}
  \noindent
  If you require any auxiliary aids, services, or other accommodations for this class, please identify your needs to your instructor by the end of the first week of class (via email or schedule an appointment in person), or as soon as you have the required documentation.
  \par
  If you wish to have special accommodations due to religious holidays, please request those accommodations by the end of the second week of class. Requests made after these deadlines may not be possible to honor. 

  \section*{Principles}
  \noindent
  Students in this class are encouraged to speak up and participate during class meetings. Because the class will represent a diversity of individual beliefs, backgrounds, and experiences, every member of this class must show respect for every other member of this class.
  \par
  I am part of the Safe Zone Ally community network of trained Chico State faculty/staff/students who are available to listen and support you in a safe and confidential manner. As a Safe Zone Ally, I can help you connect with resources on campus to address problems you may face that interfere with your academic and social success on campus as it relates to issues surrounding sexual orientation/gender identity. My goal is to help you be successful and to maintain a safe and equitable campus.

\end{document}
