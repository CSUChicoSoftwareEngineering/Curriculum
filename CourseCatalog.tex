% Maintained at https://github.com/CSUChicoSoftwareEngineering/Curriculum
\documentclass[12pt]{article}

\newlength\tindent
\setlength{\tindent}{\parindent}
\setlength{\parindent}{0pt}
\renewcommand{\indent}{\hspace*{\tindent}}

\title{CSCI Course Descriptions}
\author{Kevin Buffardi}
\date{October 2014}
\begin{document}
  \maketitle
  \section*{CSCI 430 Software Engineering}	 	
  \textit{3.0 FA WP} \\
  \underline{Prerequisites}: CSCI 311 for CSCI/CINS/APCG majors or EECE 337 for Engineering majors; ENGL 130 or JOUR 130 (or equivalent) all with a grade of C- or higher.

  An overview of software engineering principles, practices, and tools. Topics include: agile software engineering methodologies, requirements engineering, test-driven development, software design patterns, MVC architecture, version control, software metrics, and static analysis. Students work in groups to design and implement a semester-long open source software project. 

  \textit{2 hours discussion, 2 hours activity. This is an approved Writing Proficiency course; a grade of C- or better certifies writing proficiency for majors.}
  
  \section*{CSCI 431 Usability Engineering}
  \textit{3.0 SP} \\
  \underline{Prerequisites}: CINS 110 or CINS 465 for CSCI/CINS majors; CDES 322, CDES 327, CDES 437 or APCG 360 for other majors.

  An in-depth study of user experience (UX) design with an emphasis on usability evaluation methods. Students practice hands-on techniques including: usability testing, survey design, card sorting, contextual inquiry, wireframing and rapid prototyping. Students work in multi-disciplinary teams on user experience design projects.

  \textit{2 hours discussion, 2 hours activity.}

 % This *may* change to "Software Testing and Quality Assurance" or that may be added as a new course
 % \section*{CSCI 630 Software Engineering}
 % \textit{3.0 SP} \\
 % \underline{Prerequisites}: CSCI 430 and classified graduate standing.

  % In-depth study and application of the planning, design, implementation, and management of complex software systems. Topics include requirements engineering, formal specifications, object-oriented analysis, design patterns, and peopleware. Teams of students will implement a large software project using a cutting edge software engineering approach. 

  % \textit{3 hours discussion.}
\end{document}