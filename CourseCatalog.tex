\documentclass[12pt]{article}

\newlength\tindent
\setlength{\tindent}{\parindent}
\setlength{\parindent}{0pt}
\renewcommand{\indent}{\hspace*{\tindent}}

\title{CSCI Course Descriptions}
\author{Kevin Buffardi}
\date{October 2014}
\begin{document}
  \maketitle
  \section*{CSCI 430 Software Engineering}	 	
  \textit{3.0 FA WP}
  
  \underline{Prerequisites}: CSCI 311 for CSCI/CINS/APCG majors or EECE 337 for Engineering majors; ENGL 130 or JOUR 130 (or equivalent) all with a grade of C- or higher.

  An overview of software engineering principles and practice. Topics include: traditional software engineering methodologies, agile software engineering methodologies, requirements engineering, software design, risk analysis, quality assurance, testing, group dynamics, communication, and project planning/management. Students work in groups to design and implement a semester long software project. 

  \textit{2 hours discussion, 2 hours activity. This is an approved Writing Proficiency course; a grade of C- or better certifies writing proficiency for majors.}
  
  \section*{CSCI 431 Software Engineering Tools}
  \textit{3.0 SP}	 

  \underline{Prerequisites}: CSCI 430 with a grade of C- or higher.

  An in depth look at software development tools and software engineering methodology. Topics include: agile software development, version control, static and dynamic code analysis, bug tracking, debugging, and build management. Students work in groups on a semester long project to understand and modify and existing large open source product. An agile software engineering methodology is used to manage the modification project. 

  \textit{2 hours discussion, 2 hours activity.}

  \section*{CSCI 630	Software Engineering}
  \textit{3.0 SP}
  \underline{Prerequisites}: CSCI 430 and classified graduate standing.

  In-depth study and application of the planning, design, implementation, and management of complex software systems. Topics include requirements engineering, formal specifications, object-oriented analysis, design patterns, and peopleware. Teams of students will implement a large software project using a cutting edge software engineering approach. 

  \textit{3 hours discussion.}
\end{document}