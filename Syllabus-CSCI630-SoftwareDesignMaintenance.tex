\documentclass[12pt]{article}
\usepackage{hyperref}

\newlength\tindent
\setlength{\tindent}{\parindent}
\setlength{\parindent}{0pt}
\renewcommand{\indent}{\hspace*{\tindent}}
\sloppy

\title{Syllabus: CSCI 630 Software Design \& Maintenance}
\author{Kevin Buffardi}
\date{October 2018}
\begin{document}
  \maketitle
  \subsection*{Overview}	 
  \noindent
  A study of designing and maintaining complex software. The course builds upon fundamental software engineering skills with an emphasis on: object-oriented software design patterns, anti-patterns, code review and refactoring, and tools for evaluating code quality. Students practice maintaining software by collaborating on a large-scale open source project using automated development operation (DevOps) tools.

  \underline{Prerequisites}: CSCI 430 and classified graduate standing.
  
  \subsection*{Instructor}
  \noindent
  Kevin Buffardi, kbuffardi@csuchico.edu, Office hours: To be annouced.

  \subsection*{Required Materials}
  \noindent
  Laptop computer; No textbook required

  Suggested reading: 
  
  \smallskip
  "Design Patterns: Elements of Reusable Object-Oriented Software"\\
  (Gamma, Helm, Johnson, Vlissides)\\
  ISBN: 0-201-63361-2
  
  \smallskip
  "The Mythical Man-Month: Essays on Software Engineering"\\
  (Brooks)\\
  ISBN: 0-201-83595-9

  \subsection*{Schedule}
  \noindent
  This is the tentative semester schedule, subject to change.

  \begin{enumerate}
    \item Course \& project introduction, accelerated review of advanced version control (\textit{Git \& GitHub})
    \item Interfaces \& Advanced Object-Oriented Design principles (\textit{Java})
    \item Introduction to design patterns and anti-patterns
    \item Composition pattern designs
    \item Implementing Composite, Adapter, \& Decorator patterns
    \item Creational pattern designs
    \item Implementing Singleton \& abstract factory patterns
    \item Behavioral pattern designs
    \item Implementing Iterator, Observer, \& Strategy patterns
    \item Design Patterns review and exam
    \item Bug tracking, code review, \& refactoring (\textit{Java, GitHub})
    \item Accelerated review of unit testing (\textit{JUnit})
    \item Build Automation \& Continuous Integration (\textit{Gradle, Jenkins})
    \item Static and Coverage Analysis (\textit{PMD, Cobertura})
    \item Project review
  \end{enumerate}

  \subsection*{Learning Outcomes}
  \noindent
  Learning comes in different forms. From this course, students are expected to \textit{minimally} gain the following learning outcomes, with \textbf{Core Body of Knowledge} topics from \href{https://www.acm.org/binaries/content/assets/education/gsew2009.pdf}{ACM Curriculum Guidelines for Graduate Degree Programs in Software Engineering}

  \begin{itemize}
    \item \textit{Comprehension and Application} of \textbf{Software Design Fundamentals} including: general design concepts; context of software design; and software design process
    \item \textit{Application} of \textbf{Key Issues in Software Design} including: distribution of components; interaction and presentation; and data persistence
    \item \textit{Application and Analysis} of \textbf{Software Structure and Architecture} including architectural styles (macro architectural patterns) and design patterns (micro architectural patterns) 
    \item \textit{Application} of \textbf{Software Design Quality Analysis and Evaluation} including: quality attributes; quality analysis and evaluation techniques; and measures
    \item \textit{Application} of \textbf{Software Design Notations} including both structural (static) descriptions and behavioral (dynamic) descriptions
    \item \textit{Application and Analysis} of \textbf{Software Design Strategies and Methods} with an emphasis on Object-oriented design
    \item \textit{Application and Analysis} of \textbf{Testing Techniques} with a focus on unit testing and \textbf{Test-Related Measures} by evaluation of tests with code coverage
    \item \textit{Comprehension} of \textbf{Software Maintenance Fundamentals} including: definitions and terminology; nature of maintenance; and need for maintenance
    \item \textit{Application} of \textbf{Key Issues in Software Maintenance, the Maintenance Process, and Techniques for Maintenance} including: technical and management issues; software maintenance measurement; maintenance activities; program comprehension; and reengineering 
  \end{itemize}

  These outcomes are categorized according to Bloom's Taxonomy of the Cognitive domain, as \textit{italicized}.



\end{document}