\documentclass[12pt]{article}
\usepackage{hyperref}

\sloppy

\title{Syllabus: CSCI 630 Software Design \& Maintenance}
\author{Kevin Buffardi}
\date{October 2018}
\begin{document}
  \maketitle
  \subsection*{Overview}	 
  \noindent
  A study of designing and maintaining complex software. The course builds upon fundamental software engineering skills with an emphasis on: object-oriented software design patterns, anti-patterns, code review and refactoring, and tools for evaluating code quality. Students practice maintaining software by collaborating on a large-scale open source project using automated development operation (DevOps) tools.

  \underline{Prerequisites}: CSCI 430 and classified graduate standing.
  
  \subsection*{Instructor}
  \noindent
  Kevin Buffardi, kbuffardi@csuchico.edu, Office hours: To be annouced.

  \subsection*{Required Materials}
  \noindent
  Laptop computer; No textbook required
  \noindent
  Suggested reading: 
  
  \smallskip
  \noindent
  "Design Patterns: Elements of Reusable Object-Oriented Software"\\
  (Gamma, Helm, Johnson, Vlissides)\\
  ISBN: 0-201-63361-2
  
  \smallskip
  \noindent
  "The Mythical Man-Month: Essays on Software Engineering"\\
  (Brooks)\\
  ISBN: 0-201-83595-9

  \subsection*{Schedule}
  \noindent
  This is the tentative semester schedule, subject to change.

  \begin{enumerate}
    \item Course \& project introduction, accelerated review of advanced version control (\textit{Git \& GitHub})
    \item Interfaces \& Advanced Object-Oriented Design principles (\textit{Java})
    \item Introduction to design patterns and anti-patterns
    \item Composition pattern designs
    \item Implementing Composite, Adapter, \& Decorator patterns
    \item Creational pattern designs
    \item Implementing Singleton \& abstract factory patterns
    \item Behavioral pattern designs
    \item Implementing Iterator, Observer, \& Strategy patterns
    \item Design Patterns review and exam
    \item Bug tracking, code review, \& refactoring (\textit{Java, GitHub})
    \item Accelerated review of unit testing (\textit{JUnit})
    \item Build Automation \& Continuous Integration (\textit{Gradle, Jenkins})
    \item Static and Coverage Analysis (\textit{PMD, Cobertura})
    \item Project review
  \end{enumerate}

  \newpage
  \subsection*{Learning Outcomes}
  \noindent
  Learning comes in different forms. From this course, students are expected to \textit{minimally} gain the following learning outcomes, with \textbf{Core Body of Knowledge} topics from \href{https://www.acm.org/binaries/content/assets/education/gsew2009.pdf}{ACM Curriculum Guidelines for Graduate Degree Programs in Software Engineering}

  \begin{itemize}
    \item \textit{Comprehension and Application} of \textbf{Software Design Fundamentals} including: general design concepts; context of software design; and software design process
    \item \textit{Application} of \textbf{Key Issues in Software Design} including: distribution of components; interaction and presentation; and data persistence
    \item \textit{Application and Analysis} of \textbf{Software Structure and Architecture} including architectural styles (macro architectural patterns) and design patterns (micro architectural patterns) 
    \item \textit{Application} of \textbf{Software Design Quality Analysis and Evaluation} including: quality attributes; quality analysis and evaluation techniques; and measures
    \item \textit{Application} of \textbf{Software Design Notations} including both structural (static) descriptions and behavioral (dynamic) descriptions
    \item \textit{Application and Analysis} of \textbf{Software Design Strategies and Methods} with an emphasis on Object-oriented design
    \item \textit{Application and Analysis} of \textbf{Testing Techniques} with a focus on unit testing and \textbf{Test-Related Measures} by evaluation of tests with code coverage
    \item \textit{Comprehension} of \textbf{Software Maintenance Fundamentals} including: definitions and terminology; nature of maintenance; and need for maintenance
    \item \textit{Application} of \textbf{Key Issues in Software Maintenance, the Maintenance Process, and Techniques for Maintenance} including: technical and management issues; software maintenance measurement; maintenance activities; program comprehension; and reengineering 
  \end{itemize}

  \noindent 
  These outcomes are categorized according to Bloom's Taxonomy of the Cognitive domain, as \textit{italicized}.

  \newpage
  \subsection*{Grades}

  Letter grades (A through F) follow a traditional, descending 10-point scale, where the top 2 points earn a + and bottom 2 points earn a - for each letter grade. There is no extra credit nor grade inflation. Final grades are calculated with the following weights:

  \textbf{40\%} Open Source maintenance portfolio
  
  \textbf{20\%} Design pattern implementation projects
  
  \textbf{20\%} Exercises (learning activity participation and completion)

  \textbf{10\%} Midterm

  \textbf{10\%} Personal Introduction

  \subsection*{Intellectual Integrity Policy}

  We highly value learning and hard work. Therefore, in accordance with department policy, this class has a strict policy for any instances of cheating, which includes plagiarism, unauthorized collaboration, or otherwise misrepresenting someone else's thoughts as your own.

  Software Engineering is an important and valuable skill for the 21st century and it is our responsibility that grades reflect those earned demonstrating your own understanding of the subject. Holding ourselves to any lower standard hurts the reputation of the department, school, and misrepresents a skill set that can result in real harm to society.

  Any graded work should be considered an individual assignment, unless stated explicitly otherwise by the instructor. On individual assignments, students may not either give nor receive help from any other parties or resources. When reference material is allowed on a graded assignment, students are expected to properly cite their sources and avoid plagiarism and give credit to the original author's intellectual contributions. When there is any question or doubt about a source or interaction is allowed, always consult the instructor first.

  If you make the poor choice of breaking any of the intellectual integrity policies, the repercussions include:

  \begin{itemize}
    \item 0 for the assignment that can not be dropped nor made up. 
    \item Referral to Student Judicial Affairs, who decide appropriate action from the school including probation or up to expulsion
    \item For egregious or repeated offenses, F for the course. Courses failed due to intellectual integrity violations cannot be re-taken for forgiveness
  \end{itemize}

  \section*{Accommodations}

  If you need any special accommodations, contact the instructor outside of class so arrangements can be made.

  \medskip
  \textbf{Confidentiality and Mandatory Reporting}

  \smallskip
  As an instructor, one of my responsibilities is to help create a safe learning environment on our campus.  I also have a mandatory reporting responsibility related to my role as a professor.  I am required to share information regarding sexual misconduct with the University. Students may speak to someone confidentially by contacting the Counseling and Wellness Center (898-6345) or Safe Place (898-3030).  Information on campus reporting obligations and other Title IX related resources are available here: www.csuchico.edu/title-ix

  \medskip
  \textbf{Inclusion and Respect}

  \smallskip
  As an educator, I take it as my responsibility to make the classroom an effective learning environment. I expect everyone to be a part of a respectful environment both inside and outside of class. At Chico State, everyone should feel welcome and respected. If you have concerns, please bring them to my attention via email (kbuffardi@csuchico.edu) or in an appointment in my office (OCNL 220). I am trained as a Safe Zone Ally and Dreamer Ally. However, if you do not feel comfortable talking to me, I also recommend reaching out to the Office of Diversity and Inclusion.

  \section*{Professional Work Policy}

  Just in life, there is no extra credit. However, there are opportunities to either make up for small mistakes or magnify them.

  By demonstrating outstanding work, every student has the opportunity to have poor grades forgiven (exempt from grade calculation, as a 0 out of 0). Outstanding Work Credit (\textbf{OWC}) is awarded at the instructor's discretion to acknowledge work that demonstrates critical thinking and/or hard work. OWC may be awarded for excellent questions, volunteering, or otherwise going above and beyond. For each OWC earned, at the end of the semester that student's lowest exercise grade will be dropped. For example, if three OWC have been earned, the three lowest grades will be dropped.

  On the other hand, Preschool Marks (\textbf{PM}) are also given at the instructor's discretion to formally acknowledge when a student does not live up to expectations. For example, PM may be given for repeatedly showing up late to class, forgetting required materials, distracting yourself or others with phone use, or otherwise demonstrating irresponsibility; in other words, acting more like a preschooler than a responsible college student. Each PM cancels out an OWC. A net negative of OWC-PM results in downgrading the student's final letter grade by one step per net negative mark. For example, if the student earned a 95\% \textit{A} but received one more PM than OWC, then the letter grade assigned is \textit{A-}. Two more PM than OWC would result in \textit{B+}, three more would result in \textit{B}, and so on.

  Additionally, in accordance with university rules, the instructor reserves the right to drop a student from the class if they miss more than two class meetings without notifying the instructor of a valid excuse (such as health or family emergencies) before the meeting (or as soon as possible when unanticipated emergencies arise).

  \section*{Late Work Policy}
  Late submission of assignments results in 1 PM per 24-hour period late. For example, turning in an assignment 25 hours late earns 2 PM. Exceptions are only made when extensions are granted (at the instructor's discretion) before the deadline and/or when there is a documented personal or family emergency.
\end{document}