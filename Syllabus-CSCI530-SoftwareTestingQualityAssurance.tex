\documentclass[12pt]{article}

\newlength\tindent
\setlength{\tindent}{\parindent}
\setlength{\parindent}{0pt}
\renewcommand{\indent}{\hspace*{\tindent}}

\title{Syllabus: CSCI 530 Software Testing and Quality Assurance}
\author{Kevin Buffardi}
\date{April 2015}
\begin{document}
  \maketitle
  \section*{Overview}	 
  \noindent
  In-depth study of software verification and validation with a concentration on software testing methods, tools, and metrics. Topics include Test-Driven Development, unit testing, integration testing, acceptance testing, as well as software and testing metrics. Students will work individually as well as in teams to verify and validate large open source projects.
  \newline
  \newline
  \underline{Prerequisites}: CSCI 430 or CSCI 630. \\
  
  \section*{Instructor}
  \noindent
  Kevin Buffardi \\
  kbuffardi@csuchico.edu \\
  OCNL 220 \\
  Office Hours: TBD\\

  \section*{Required Materials}
  \noindent
  Textbook: Software Testing: A Craftsman's Approach (Paul C. Jorgensen)
  \newline
  Laptop computer (with Mac OSX or *nix operating system or virtual machine) required in class.

  \section*{Learning Outcomes}
  \noindent
  By completing this course, students will be able to:
  \begin{itemize}
    \item Understand and explain the purpose and impact of software testing
    \item Demonstrate contemporary testing approaches (such as Test-Driven and Behavior-Driven Development)
    \item Analyze algorithms and identifying appropriate testing approaches
    \item Create and execute test plans
    \item Evaluate the quality of existing tests
  \end{itemize}

  The course addresses following broader CSCI Learning Outcomes: \\
  \textit{\textbf{I}ntroduces, \textbf{P}racticed, \textbf{A}ssessed} \\
  \begin{itemize}
    \item[a.] An ability to apply knowledge of computing and mathematics appropriate to the discipline. \textbf{(P)}
    \item[b.]  An ability to analyze a problem, and identify and define the computing requirements appropriate to its solution. \textbf{(A)}
    \item[c.] An ability to design, implement, and evaluate a computer-based system, process, component, or program to meet desired needs. \textbf{(A)}
    \item[d.] An ability to function effectively on teams to accomplish a common goal. \textbf{(A)}
    \item[e.] An understanding of professional, ethical, legal, security and social issues and responsibilities. \textbf{(P)}
    \item[f.] An ability to communicate effectively with a range of audiences. \textbf{(P)}
    \item[g.] An ability to analyze the local and global impact of computing on individuals, organizations, and society. \textbf{(P)}
    \item[h.] Recognition of the need for and an ability to engage in continuing professional development. \textbf{(P)}
    \item[i.] An ability to use current techniques, skills, and tools necessary for computing practice. \textbf{(A)}
    \item[j.] An ability to apply mathematical foundations, algorithmic principles, and computer science theory in the modeling and design of computer-based systems in a way that demonstrates comprehension of the tradeoffs involved in design choices. \textbf{(A)} 
    \item[k.] An ability to apply design and development principles in the construction of software systems of varying complexity. \textbf{(A)}
  \end{itemize}

  \section*{Assessment}
  \noindent
  Grades for this course will be determined by the following assessments, with the provided weights:
  \begin{itemize}
    \item Team Project Deliverables \hfill \\
      \begin{itemize}
        \item Open Source Project Testing \textbf{(20\%)}
        \item Testing Evaluation \textbf{(20\%)}
      \end{itemize}
    \item Individual Assignments: \hfill \\
      \begin{itemize}
        \item In-class exercises \textbf{(25\%, cumulative)}
        \item Mid-Term exam \textbf{(15\%)}
        \item Final exam \textbf{(20\%)}
      \end{itemize}
  \end{itemize}
  
  \section*{Schedule}
  \noindent
  The following is a \underline{rough schedule} of the semester, which is \underline{subject to change} based on the instructor's discretion: \\
  \newline
  \textbf{Week} - Topic 
  \begin{enumerate}
    \item - Overview of Software Testing
    \item - Unit Testing
    \item - Test-Driven Development
    \item - Acceptance Testing
    \item - Behavior-Driven Development
    \item - Decision Table-Based Testing
    \item - Boundary Value Testing
    \item - Path Testing
    \item - Data Flow Testing
    \item - Testing Metrics
    \item - All Pairs Testing
    \item - Mutation Testing
    \item - Integration Testing
    \item - System Testing
    \item - Comprehensive course review
  \end{enumerate}

  \section*{Accommodations}
  \noindent
  If you require any auxiliary aids, services, or other accommodations for this class, please identify your needs to your instructor by the end of the first week of class (via email or schedule an appointment in person), or as soon as you have the required documentation.\\ 
  If you wish to have special accommodations due to religious holidays, please request those accommodations by the end of the second week of class. Requests made after these deadlines may not be possible to honor. 

  \section*{Principles}
  \noindent
  Respect: Students in this class are encouraged to speak up and participate during class meetings. Because the class will represent a diversity of individual beliefs, backgrounds, and experiences, every member of this class must show respect for every other member of this class. \\
  \newline
  I am part of the Safe Zone Ally community network of trained Chico State faculty/staff/students who are available to listen and support you in a safe and confidential manner. As a Safe Zone Ally, I can help you connect with resources on campus to address problems you may face that interfere with your academic and social success on campus as it relates to issues surrounding sexual orientation/gender identity. My goal is to help you be successful and to maintain a safe and equitable campus.
\end{document}